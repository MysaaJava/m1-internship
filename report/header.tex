% Loading packages 
\usepackage{autofe}
\usepackage{hyperref}
\hypersetup{
	colorlinks=true,
	linkcolor=blue,
	filecolor=magenta,      
	urlcolor=cyan
}
\usepackage{amsmath}
\usepackage{amssymb}
\usepackage{bbm}
\usepackage{stmaryrd}
\usepackage[main=english]{babel}
\usepackage{csquotes}
\usepackage{listings}
\usepackage{lstautogobble}
\usepackage{svg}
\usepackage{tikz}
\usepackage{multirow}
\usepackage{multicol}
\usepackage{tcolorbox}
\usepackage{mdframed}
\usepackage{proof}
\usepackage{biblatex}
\usepackage{xparse}
\usepackage{cprotect}
\usepackage{titlesec}
\usepackage{xpatch}
\usepackage{enumitem}
\usepackage{amsfonts}
\usepackage{mathtools}
\usepackage[page,header]{appendix}
\usepackage{minitoc}
\usepackage{mathtools}
\usepackage{textgreek}

\usepackage{geometry}

\usepackage{csquotes}
\usepackage[lighttt]{lmodern}
\usetikzlibrary{shapes.geometric,positioning}

% Macros caractères globales
\newcommand{\pgrph}{\P}
\newcommand{\hsep}{\vspace{.2cm}\centerline{\rule{0.8\linewidth}{.05pt}}\vspace{.4cm}}
\renewcommand{\P}{\mathbb{P}}
\newcommand{\E}{\mathbb{E}}
\newcommand{\1}{\scalebox{1.2}{$\mathbbm{1}$}}
\newcommand{\floor}[1]{\left\lfloor#1\right\rfloor}
\newcommand{\littleO}{o}
\newcommand{\bigO}{\mathcal{O}}
\newcommand{\longdash}{\:\textrm{---}\:}
\newcommand\hole{\left[\raisebox{-0.25ex}{\scalebox{1.2}{$\cdot$}}\right]}
\newcommand\bracket[1]{\!\left[#1\right]}
\newcommand{\ssi}{\quad\text{\underline{ssi}}\quad}
\newcommand\eng[1]{\textit{\foreignlanguage{english}{#1}}}
\newcommand\spacebar{\;|\;}
\def\nDownarrow{\not\mspace{1mu}\Downarrow}
\let\pprec\preccurlyeq


% Création des environnement globaux
\newtheorem{theorem}{Théorème}
\newtheorem{definition}{Definition}
\newtheorem{property}{Propriété}

\newcounter{rule}
\addto\extrasfrench{%
	\renewcommand{\figureautorefname}{\textsc{Figure}}
	\renewcommand{\sectionautorefname}{Section}
	\renewcommand{\subsectionautorefname}{Sous-section}
	\renewcommand{\appendixautorefname}{Annexe}
	\renewcommand{\theoremautorefname}{Théorème}
	\providecommand\propertyautorefname{Propriété}
	\providecommand\ruleautorefname{règle}
}

% Commandes logiques globales
\newcommand{\ifnullthenelse}[3]{
	\ifnum\value{#1}=0
	#2
	\else
	#3
	\fi
}

%%% Commande \newtag permettant de changer le label d'une equation
\makeatletter
\newcommand\newtag[2]{#1\def\@currentlabel{#1}\label{#2}}
\makeatother


% Macros caractères spécifiques au document
\newcommand\Tm{\ensuremath{\operatorname{Tm}}}
\newcommand\Set{\AgdaPrimitive{Set}}
\newcommand\For{\ensuremath{\operatorname{For}}}
\newcommand\Prop{\AgdaPrimitive{Prop}}
\newcommand\R{\operatorname{R}}
\newcommand\lam{\ensuremath{\operatorname{lam}}}
\newcommand\app{\operatorname{app}}
\newcommand\foralli{\operatorname{\operatorname{\forall i}}}
\newcommand\foralle{\operatorname{\operatorname{\forall e}}}
\newcommand\Pf{\ensuremath{\operatorname{Pf}\;}}
\newcommand\bCon{\textbf{Con}}
\newcommand\bSet{\textbf{Set}}
\newcommand\bProp{\textbf{Prop}}

% Agda Config
\usepackage{agda}
%\AgdaNoSpaceAroundCode{}
\usepackage{newunicodechar}
\newunicodechar{∘}{\ensuremath{\mathnormal{\circ}}}
\newunicodechar{≡}{\ensuremath{\mathnormal{\equiv}}}
\newunicodechar{◇}{\ensuremath{\mathnormal{\diamond}}}
\newunicodechar{Γ}{\textGamma}
\newunicodechar{Δ}{\textDelta}
\newunicodechar{Ξ}{\textXi}
\newunicodechar{α}{\textalpha}
\newunicodechar{β}{\textbeta}
\newunicodechar{γ}{\textgamma}
\newunicodechar{δ}{\textdelta}
\newunicodechar{ε}{\textvarepsilon}
\newunicodechar{σ}{\textsigma}
\newunicodechar{ι}{\textiota}
\newunicodechar{π}{\textpi}
\newunicodechar{λ}{\textlambda}
\newunicodechar{ℓ}{\ensuremath{\ell}}
\newunicodechar{▹}{\ensuremath{\mathnormal{\triangleright}}}
\newunicodechar{⊢}{\ensuremath{\mathnormal{\vdash}}}
\newunicodechar{⇒}{\ensuremath{\mathnormal{\Rightarrow}}}
\newunicodechar{∀}{\ensuremath{\mathnormal{\forall}}}
\newunicodechar{≈}{\ensuremath{\mathnormal{\approx}}}
\newunicodechar{ₜ}{\ensuremath{{}_\text{t}}}
\newunicodechar{ₚ}{\ensuremath{{}_\text{p}}}
\newunicodechar{⁰}{\ensuremath{{}^0}}


\newcommand\agda[1]{
	\small
	\begin{code}
		\input{#1}
	\end{code}
}
% Création des environnements spécifiques au document


%%% Subparaghaphs box
\newtcbox{\subparaghaphbox}{nobeforeafter,tcbox raise base, arc=9pt, outer arc=9pt, boxsep=2pt,left=2pt,right=2pt,top=2pt,bottom=2pt,boxrule=1pt,colback=white!85!orange}
\newcommand{\subparaghaphboxedcontent}[1]{\subparaghaphbox{#1}\newline}
%\titleclass{\mathcases}{straight}[\subparagraph]
\titleformat{\subparagraph}[runin]{\normalfont\normalsize\bfseries}{}{0em}{\subparaghaphboxedcontent}
\titlespacing*{\subparagraph}{0pt}{3.25ex plus 1ex minus .2ex}{0.5em}


\addbibresource{Bilibibio.bib}
