% !TeX spellcheck = en_US
\documentclass[10pt,a4paper]{article}

% Loading packages 
\usepackage{autofe}
\usepackage{hyperref}
\hypersetup{
	colorlinks=true,
	linkcolor=blue,
	filecolor=magenta,      
	urlcolor=cyan
}
\usepackage{amsmath}
\usepackage{amssymb}
\usepackage{bbm}
\usepackage{stmaryrd}
\usepackage[main=english]{babel}
\usepackage{csquotes}
\usepackage{listings}
\usepackage{lstautogobble}
\usepackage{svg}
\usepackage{tikz}
\usepackage{multirow}
\usepackage{multicol}
\usepackage{tcolorbox}
\usepackage{mdframed}
\usepackage{proof}
\usepackage{biblatex}
\usepackage{xparse}
\usepackage{cprotect}
\usepackage{titlesec}
\usepackage{xpatch}
\usepackage{enumitem}
\usepackage{amsfonts}
\usepackage{mathtools}
\usepackage[page,header]{appendix}
\usepackage{minitoc}
\usepackage{mathtools}

\usepackage{geometry}

\usepackage{csquotes}
\usepackage[lighttt]{lmodern}
\usetikzlibrary{shapes.geometric,positioning}

% Macros caractères globales
\newcommand{\pgrph}{\P}
\newcommand{\hsep}{\vspace{.2cm}\centerline{\rule{0.8\linewidth}{.05pt}}\vspace{.4cm}}
\renewcommand{\P}{\mathbb{P}}
\newcommand{\E}{\mathbb{E}}
\newcommand{\1}{\scalebox{1.2}{$\mathbbm{1}$}}
\newcommand{\floor}[1]{\left\lfloor#1\right\rfloor}
\newcommand{\littleO}{o}
\newcommand{\bigO}{\mathcal{O}}
\newcommand{\longdash}{\:\textrm{---}\:}
\newcommand\hole{\left[\raisebox{-0.25ex}{\scalebox{1.2}{$\cdot$}}\right]}
\newcommand\bracket[1]{\!\left[#1\right]}
\newcommand{\ssi}{\quad\text{\underline{ssi}}\quad}
\newcommand\eng[1]{\textit{\foreignlanguage{english}{#1}}}
\newcommand\spacebar{\;|\;}
\def\nDownarrow{\not\mspace{1mu}\Downarrow}
\let\pprec\preccurlyeq


% Création des environnement globaux
\newtheorem{theorem}{Théorème}
\newtheorem{definition}{Definition}
\newtheorem{property}{Propriété}

\newcounter{rule}
\addto\extrasfrench{%
	\renewcommand{\figureautorefname}{\textsc{Figure}}
	\renewcommand{\sectionautorefname}{Section}
	\renewcommand{\subsectionautorefname}{Sous-section}
	\renewcommand{\appendixautorefname}{Annexe}
	\renewcommand{\theoremautorefname}{Théorème}
	\providecommand\propertyautorefname{Propriété}
	\providecommand\ruleautorefname{règle}
}

% Commandes logiques globales
\newcommand{\ifnullthenelse}[3]{
	\ifnum\value{#1}=0
	#2
	\else
	#3
	\fi
}

%%% Commande \newtag permettant de changer le label d'une equation
\makeatletter
\newcommand\newtag[2]{#1\def\@currentlabel{#1}\label{#2}}
\makeatother


% Macros caractères spécifiques au document
\newcommand\Tm{\operatorname{Tm}}
\newcommand\Set{\operatorname{Set}}
\newcommand\For{\operatorname{For}}
\newcommand\Prop{\operatorname{Prop}}
\newcommand\R{\operatorname{R}}
\newcommand\lam{\operatorname{lam}}
\newcommand\app{\operatorname{app}}
\newcommand\foralli{\operatorname{\operatorname{\forall i}}}
\newcommand\foralle{\operatorname{\operatorname{\forall e}}}
\newcommand\Pf{\operatorname{Pf}\;}
\newcommand\bCon{\textbf{Con}}
\newcommand\bSet{\textbf{Set}}
\newcommand\bProp{\textbf{Prop}}

% Agda Config
\usepackage{agda}
\usepackage{newunicodechar}
\newunicodechar{∘}{\ensuremath{\mathnormal{\circ}}}
\newunicodechar{≡}{\ensuremath{\mathnormal{\equiv}}}
\newunicodechar{◇}{\ensuremath{\mathnormal{\diamond}}}
\newunicodechar{Γ}{\ensuremath{\mathnormal{\Gamma}}}
\newunicodechar{Δ}{\ensuremath{\mathnormal{\Delta}}}
\newunicodechar{Ξ}{\ensuremath{\mathnormal{\Xi}}}
\newunicodechar{α}{\ensuremath{\mathnormal{\alpha}}}
\newunicodechar{β}{\ensuremath{\mathnormal{\beta}}}
\newunicodechar{γ}{\ensuremath{\mathnormal{\gamma}}}
\newunicodechar{δ}{\ensuremath{\mathnormal{\delta}}}
\newunicodechar{ε}{\ensuremath{\mathnormal{\varepsilon}}}
\newunicodechar{σ}{\ensuremath{\mathnormal{\sigma}}}
\newunicodechar{π}{\ensuremath{\mathnormal{\pi}}}
\newunicodechar{λ}{\ensuremath{\mathnormal{\lambda}}}
\newunicodechar{▹}{\ensuremath{\mathnormal{\triangleright}}}
\newunicodechar{⊢}{\ensuremath{\mathnormal{\vdash}}}
\newunicodechar{⇒}{\ensuremath{\mathnormal{\Rightarrow}}}
\newunicodechar{∀}{\ensuremath{\mathnormal{\forall}}}
\newunicodechar{≈}{\ensuremath{\mathnormal{\approx}}}
\newunicodechar{ₜ}{\ensuremath{{}_t}}
\newunicodechar{ₚ}{\ensuremath{{}_p}}
\newunicodechar{⁰}{\ensuremath{{}^0}}


\newcommand\agda[1]{
	\small
	\begin{code}
		\input{#1}
	\end{code}
}
% Création des environnements spécifiques au document


%%% Subparaghaphs box
\newtcbox{\subparaghaphbox}{nobeforeafter,tcbox raise base, arc=9pt, outer arc=9pt, boxsep=2pt,left=2pt,right=2pt,top=2pt,bottom=2pt,boxrule=1pt,colback=white!85!orange}
\newcommand{\subparaghaphboxedcontent}[1]{\subparaghaphbox{#1}\newline}
%\titleclass{\mathcases}{straight}[\subparagraph]
\titleformat{\subparagraph}[runin]{\normalfont\normalsize\bfseries}{}{0em}{\subparaghaphboxedcontent}
\titlespacing*{\subparagraph}{0pt}{3.25ex plus 1ex minus .2ex}{0.5em}


\addbibresource{Bilibibio.bib}


\title{Completeness Proof for different logical frameworks
	\\[1ex] \large Notes on my 3 month internship at Faculty of Informatics (ELTE, Budapest)}
\hypersetup{pdftitle={Completeness Proof for different kinds of logical frameworks}}
\author{Samy Avrillon, supervised by
	\\[1ex] Ambrus Kaposi (ELTE, Budapest)
	\\[1ex] and Thorsten Altenkirsch (University of Notthingham, United Kingdom)}

\begin{document}
	
	\doparttoc
	\maketitle
	
	\hsep
	
	\tableofcontents
	
	\newpage
	
	
	\section{Introduction}
	
		In the first place, i was supposed to do this internship in Nottingham (UK) with Thorsten Altenkirsh. But, because of administrative issues, i was not able to go there, and Thorsten Altenkirsh contacted Ambrus Kaposi in Budapest, whose university agreed to accept me. Therefore, i went to Budapest and i did the internship under the physical supervision of Ambrus Kaposi, and the remote supervision of Thorsten Altenkirsch.
		
		The original subject of the internship was \enquote{Developping a simplified account of normalization by evaluation using the theory of categories with families}. But there was a lot to do, starting with learning how to use Agda.
		
		\subsection{Introduction to the problem}
			What i do call a \enquote{logic} is a set of definitions containing formulæ and a notion of provability of those formulæ, plus a set of operators/equalities to construct or reduce these rules. I have studied the most common logical frameworks, that is, Propositional Logic, First-order logic with infinitary predicates, and Predicate Logic. For each of those logics, one can define a notion of \emph{model}. A model of a certain kind of logic is something that implements all of the logic's definitions, operators and equalities. From all of those models, one can extract the \emph{initial model}, also called \emph{syntax}. It is the smallest of all models, which means that from any model of that logic, we have a morphism from the syntax to that model.
			
			Then, our goal is for each logic to prove the completeness of a specific class of models. Completeness can be stated as such: \enquote{For any formula that is \emph{true} in all models of the specified class, then the formula has a proof in the syntax}. By being true in the model, one can understand that the formula has a proof from the model.
		\subsection{Motivation}
		\subsection{Structure of this report}
	\section{A first account of Completeness and Normalization}
		\subsection{Normalization for ZOL}
		\subsection{Normalization for IFOL}
		\subsection{Merging the two proofs}
	\section{Predicate Logic}
		\subsection{SOGAT Presentation of FFOL}
		
		\begin{tcolorbox}
			\[
			\begin{array}{lcl}
				\Tm & : & \Set^+ \\
				\\
				\For & : & \Set \\
				- \implies - & : & \For \rightarrow \For \rightarrow \For\\
				\forall & : & (\Tm \rightarrow \For) \rightarrow \For \\
				\R & : & \Tm \rightarrow \Tm \rightarrow \For\\
				\\
				\Pf & : & \For \rightarrow \Prop^+ \\
				\lam & : & (\Pf A \rightarrow \Pf B) \rightarrow \Pf (A \implies B)\\
				\app & : & \Pf (A \implies B) \rightarrow (\Pf A \rightarrow \Pf B)\\
				\foralli & : & (t : \Tm \rightarrow^+ A\;t) \rightarrow \Pf (\forall A)\\
				\foralle & : & \Pf (\forall A) \rightarrow (t : \Tm) \rightarrow \Pf (A\;t)\\
			\end{array}
			\]
		\end{tcolorbox}
	
		\subsection{Turning a SOGAT Presentation into a GAT}
			\begin{tcolorbox}
				\agda{agda/Con.tex}
			\end{tcolorbox}
	
			\begin{tcolorbox}
				\agda{agda/Tm.tex}
			\end{tcolorbox}
	
			\begin{tcolorbox}
				\agda{agda/Tm+.tex}
			\end{tcolorbox}
	
			\begin{tcolorbox}
				\agda{agda/For.tex}
			\end{tcolorbox}
	
			\begin{tcolorbox}
				\agda{agda/Pf.tex}
			\end{tcolorbox}
	
			\begin{tcolorbox}
				\agda{agda/Pf+.tex}
			\end{tcolorbox}
	
			\begin{tcolorbox}
				\agda{agda/R.tex}
			\end{tcolorbox}
	
			\begin{tcolorbox}
				\agda{agda/Imp.tex}
			\end{tcolorbox}
	
			\begin{tcolorbox}
				\agda{agda/Forall.tex}
			\end{tcolorbox}
	
			\begin{tcolorbox}
				\agda{agda/LamApp.tex}
			\end{tcolorbox}
	
			\begin{tcolorbox}
				\agda{agda/ForallR.tex}
			\end{tcolorbox}
			
	\section{Implementing the Syntax}
		\subsection{Separated Contexts}
		
			\begin{tcolorbox}
				\agda{agda/ICont.tex}
			\end{tcolorbox}
		
			\begin{tcolorbox}
				\agda{agda/IFor.tex}
			\end{tcolorbox}
		
			\begin{tcolorbox}
				\agda{agda/ISubt.tex}
			\end{tcolorbox}
		
			\begin{tcolorbox}
				\agda{agda/ITm.tex}
			\end{tcolorbox}
		
			\begin{tcolorbox}
				\agda{agda/ISubtT.tex}
			\end{tcolorbox}
		
			\begin{tcolorbox}
				\agda{agda/ISubTF.tex}
			\end{tcolorbox}
		
			\begin{tcolorbox}
				\agda{agda/IIdCompT.tex}
			\end{tcolorbox}
		
			\begin{tcolorbox}
				\agda{agda/IConp.tex}
			\end{tcolorbox}
		
			\begin{tcolorbox}
				\agda{agda/ISubtC.tex}
			\end{tcolorbox}
		
			\begin{tcolorbox}
				\agda{agda/IConpTp.tex}
			\end{tcolorbox}
		
			\begin{tcolorbox}
				\agda{agda/IPf.tex}
			\end{tcolorbox}
		
			\begin{tcolorbox}
				\agda{agda/IRen.tex}
			\end{tcolorbox}
		
			\begin{tcolorbox}
				\agda{agda/ISubp.tex}
			\end{tcolorbox}
		
			\begin{tcolorbox}
				\agda{agda/ISubtP.tex}
			\end{tcolorbox}
		
			\begin{tcolorbox}
				\agda{agda/ISubtS.tex}
			\end{tcolorbox}
		
			\begin{tcolorbox}
				\agda{agda/ISubpP.tex}
			\end{tcolorbox}
		
			\begin{tcolorbox}
				\agda{agda/IIdCompP.tex}
			\end{tcolorbox}
			\begin{tcolorbox}
				\agda{agda/ICon.tex}
			\end{tcolorbox}
		
			\begin{tcolorbox}
				\agda{agda/ISub.tex}
			\end{tcolorbox}
		
			\begin{tcolorbox}
				\agda{agda/IIdComp.tex}
			\end{tcolorbox}
		
			\begin{tcolorbox}
				\agda{agda/ICExt.tex}
			\end{tcolorbox}
		
		\subsection{Transport Hell}
	
	\section{Summary}
	
	
	\section{Bibliography}
	\begingroup
	\renewcommand{\section}[2]{}%
	\printbibliography
	\endgroup
	
	\newpage
	\addappheadtotoc
	\appendix
	\addtocontents{toc}{\protect\setcounter{tocdepth}{-1}}
	\appendixpage
	
	\section{Agda Code}
	
\end{document} 
